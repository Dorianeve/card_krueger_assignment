% Options for packages loaded elsewhere
\PassOptionsToPackage{unicode}{hyperref}
\PassOptionsToPackage{hyphens}{url}
%
\documentclass[
  ignorenonframetext,
]{beamer}
\usepackage{pgfpages}
\setbeamertemplate{caption}[numbered]
\setbeamertemplate{caption label separator}{: }
\setbeamercolor{caption name}{fg=normal text.fg}
\beamertemplatenavigationsymbolsempty
% Prevent slide breaks in the middle of a paragraph
\widowpenalties 1 10000
\raggedbottom
\setbeamertemplate{part page}{
  \centering
  \begin{beamercolorbox}[sep=16pt,center]{part title}
    \usebeamerfont{part title}\insertpart\par
  \end{beamercolorbox}
}
\setbeamertemplate{section page}{
  \centering
  \begin{beamercolorbox}[sep=12pt,center]{part title}
    \usebeamerfont{section title}\insertsection\par
  \end{beamercolorbox}
}
\setbeamertemplate{subsection page}{
  \centering
  \begin{beamercolorbox}[sep=8pt,center]{part title}
    \usebeamerfont{subsection title}\insertsubsection\par
  \end{beamercolorbox}
}
\AtBeginPart{
  \frame{\partpage}
}
\AtBeginSection{
  \ifbibliography
  \else
    \frame{\sectionpage}
  \fi
}
\AtBeginSubsection{
  \frame{\subsectionpage}
}
\usepackage{lmodern}
\usepackage{amssymb,amsmath}
\usepackage{ifxetex,ifluatex}
\ifnum 0\ifxetex 1\fi\ifluatex 1\fi=0 % if pdftex
  \usepackage[T1]{fontenc}
  \usepackage[utf8]{inputenc}
  \usepackage{textcomp} % provide euro and other symbols
\else % if luatex or xetex
  \usepackage{unicode-math}
  \defaultfontfeatures{Scale=MatchLowercase}
  \defaultfontfeatures[\rmfamily]{Ligatures=TeX,Scale=1}
\fi
\usecolortheme{dolphin}
\usefonttheme{structurebold}
% Use upquote if available, for straight quotes in verbatim environments
\IfFileExists{upquote.sty}{\usepackage{upquote}}{}
\IfFileExists{microtype.sty}{% use microtype if available
  \usepackage[]{microtype}
  \UseMicrotypeSet[protrusion]{basicmath} % disable protrusion for tt fonts
}{}
\makeatletter
\@ifundefined{KOMAClassName}{% if non-KOMA class
  \IfFileExists{parskip.sty}{%
    \usepackage{parskip}
  }{% else
    \setlength{\parindent}{0pt}
    \setlength{\parskip}{6pt plus 2pt minus 1pt}}
}{% if KOMA class
  \KOMAoptions{parskip=half}}
\makeatother
\usepackage{xcolor}
\IfFileExists{xurl.sty}{\usepackage{xurl}}{} % add URL line breaks if available
\IfFileExists{bookmark.sty}{\usepackage{bookmark}}{\usepackage{hyperref}}
\hypersetup{
  pdftitle={Minimum wages and employment: A case study of the fast food industry in New Jersey and Pennsylvania},
  pdfauthor={Verena Brufatto, Claudia Manili, Luca Messina, Alessandro Villa},
  hidelinks,
  pdfcreator={LaTeX via pandoc}}
\urlstyle{same} % disable monospaced font for URLs
\newif\ifbibliography
\usepackage{longtable,booktabs}
\usepackage{caption}
% Make caption package work with longtable
\makeatletter
\def\fnum@table{\tablename~\thetable}
\makeatother
\setlength{\emergencystretch}{3em} % prevent overfull lines
\providecommand{\tightlist}{%
  \setlength{\itemsep}{0pt}\setlength{\parskip}{0pt}}
\setcounter{secnumdepth}{-\maxdimen} % remove section numbering
\usepackage{fourier}
\usepackage{array}
\usepackage{booktabs}
\usepackage{longtable}
\usepackage{array}
\usepackage{multirow}
\usepackage{wrapfig}
\usepackage{float}
\usepackage{colortbl}
\usepackage{pdflscape}
\usepackage{tabu}
\usepackage{threeparttable}
\usepackage{threeparttablex}
\usepackage[normalem]{ulem}
\usepackage{makecell}
\usepackage{xcolor}
\usepackage{caption}

\title{Minimum wages and employment: A case study of the fast food industry in
New Jersey and Pennsylvania}
\author{Verena Brufatto, Claudia Manili, Luca Messina, Alessandro Villa}
\date{oday}

\begin{document}
\frame{\titlepage}

\begin{frame}{Introduzione}
\protect\hypertarget{introduzione}{}

\begin{itemize}
\tightlist
\item
  Presentiamo l'articolo di David Card e Alan B. Kruger del settembre
  1994.
\item
  In un mercato del lavoro con bassi salari, quali sono gli effetti
  sull'occupazione dovuti ad un aumento del salario minimo?
\item
  In data 1 Aprile 1992 il salario minimo del New Jersey aumenta da
  \$4.25 a \$5.05 l'ora.
\item
  Si analizza l'industria dei fast food.
\end{itemize}

\end{frame}

\begin{frame}{Caso di Studio}
\protect\hypertarget{caso-di-studio}{}

\begin{itemize}
\tightlist
\item
  Perché l'industria del fast food?

  \begin{itemize}
  \tightlist
  \item
    Alta percentuale di lavoratori con bassi salari.
  \item
    I ristoranti rispettano la regolamentazione sul salario minimo.
  \item
    Qualifica dei lavoratori e prodotti finali omogenei, assenza di
    mance facilitano la misurazione del salario.
  \end{itemize}
\item
  Sono intervistati diversi ristoranti fast-food.
\item
  Suddivisione in due gruppi.

  \begin{itemize}
  \tightlist
  \item
    Treatment group, ristoranti del New Jersey in cui si ha l'aumento
    del salario minimo.
  \item
    Control group, ristoranti della Pennsylvania con economia
    comparabile a quella del New Jersey.
  \end{itemize}
\item
  I ristoranti sono stati intervistati prima e dopo il trattamento.
\end{itemize}

\end{frame}

\begin{frame}{Overview del campione - 1}
\protect\hypertarget{overview-del-campione---1}{}

\captionsetup[table]{labelformat=empty}
\begin{table}

\caption{\label{tab:unnamed-chunk-1}\textbf{Table 1}}
\centering
\begin{tabular}[t]{lrrr}
\toprule
  & All & NJ & PA\\
\midrule
Stores & 410 & 331 & 79\\
Refusals & 1 & 1 & 0\\
Interviewed & 399 & 321 & 78\\
Renovations & 2 & 2 & 0\\
Closed & 6 & 5 & 1\\
\bottomrule
\end{tabular}
\end{table}

\end{frame}

\begin{frame}{Overview del campione - 2}
\protect\hypertarget{overview-del-campione---2}{}

\captionsetup[table]{labelformat=empty}
\begingroup\fontsize{6}{8}\selectfont

\begin{longtable}[t]{l>{\raggedleft\arraybackslash}p{5em}>{\raggedleft\arraybackslash}p{5em}>{}p{5em}>{}p{5em}>{}p{5em}>{}p{5em}>{}p{5em}>{}p{5em}>{}p{5em}>{}p{5em}>{}p{5em}>{}p{5em}>{}p{5em}>{}p{5em}>{}p{5em}>{}p{5em}>{}p{5em}>{}p{5em}}
\caption{\label{tab:unnamed-chunk-2}\textbf{Table 2 - Means of key variables}}\\
\toprule
\textbf{ } & \textbf{NJ} & \textbf{PA}\\
\midrule
\addlinespace[0.3em]
\multicolumn{3}{l}{\textbf{Distribution of stores (\%)}}\\
\hspace{1em}Buger King & 44.30 & 41.09\\
\hspace{1em}KFC & 15.19 & 20.54\\
\hspace{1em}Roy Rogers & 21.52 & 24.77\\
\hspace{1em}Wendys & 18.99 & 13.60\\
\hspace{1em}Company-owned & 35.44 & 34.14\\
\addlinespace[0.3em]
\multicolumn{3}{l}{\textbf{Means in T1}}\\
\hspace{1em}FTE employment & 20.44 & 23.33\\
\hspace{1em}Full time employees (\%) & 32.85 & 35.04\\
\hspace{1em}Starting wage & 4.61 & 4.63\\
\hspace{1em}Hours open & 14.42 & 14.53\\
\hspace{1em}Price of meal & 3.35 & 3.04\\
\hspace{1em}Wage = 4.25\$ (\%) & 30.51 & 32.91\\
\addlinespace[0.3em]
\multicolumn{3}{l}{\textbf{Means in T2}}\\
\hspace{1em}FTE employment & 21.03 & 21.17\\
\hspace{1em}Full time employees (\%) & 35.87 & 30.38\\
\hspace{1em}Starting wage & 5.08 & 4.62\\
\hspace{1em}Hours open & 14.42 & 14.65\\
\hspace{1em}Price of meal & 3.41 & 3.03\\
\hspace{1em}Wage = 4.25\$ (\%) & 0.00 & 28.17\\
\hspace{1em}Wage = 5.05\$ (\%) & 88.99 & 1.41\\
\bottomrule
\end{longtable}
\endgroup{}

\end{frame}

\begin{frame}{Distribuzione dei salari in \(T_1\)}
\protect\hypertarget{distribuzione-dei-salari-in-t_1}{}

\hypertarget{htmlwidget-c59279f9c407cd36a22b}{}

\end{frame}

\begin{frame}{Distribuzione dei salari in \(T_2\)}
\protect\hypertarget{distribuzione-dei-salari-in-t_2}{}

\hypertarget{htmlwidget-a5853853dc1da4ffe424}{}

\hypertarget{htmlwidget-535f6bc813fdc3e0ebc2}{}

\end{frame}

\begin{frame}{Sintesi}
\protect\hypertarget{sintesi}{}

\begin{longtable}[]{@{}ll@{}}
\toprule
Variabile & Descrizione\tabularnewline
\midrule
\endhead
Treatment &
\makecell{Variazione del salario minimo \\ del New Jersey nell'aprile 1992}\tabularnewline
Outcome & Occupazione\tabularnewline
Identification & Difference in differences\tabularnewline
Treatment group & New Jersey\tabularnewline
Control group & Pennsylvania\tabularnewline
Before & Febbraio 1992\tabularnewline
After & Novembre 1992\tabularnewline
\bottomrule
\end{longtable}

\end{frame}

\begin{frame}{Difference in differences}
\protect\hypertarget{difference-in-differences}{}

\(y = \beta_0 + \beta_1 T + \beta_2 S + \beta_3 (TS) + \varepsilon\)

dove

\begin{itemize}
\tightlist
\item
  \(T = 1\) se \(t = After\)
\item
  \(S = 1\) se \(s = NJ\)
\end{itemize}

\end{frame}

\begin{frame}{Difference in differences}
\protect\hypertarget{difference-in-differences-1}{}

\begin{longtable}[]{@{}llll@{}}
\toprule
\begin{minipage}[b]{0.16\columnwidth}\raggedright
\(y_{st}\)\strut
\end{minipage} & \begin{minipage}[b]{0.19\columnwidth}\raggedright
s = 2\strut
\end{minipage} & \begin{minipage}[b]{0.19\columnwidth}\raggedright
s = 1\strut
\end{minipage} & \begin{minipage}[b]{0.35\columnwidth}\raggedright
Difference\strut
\end{minipage}\tabularnewline
\midrule
\endhead
\begin{minipage}[t]{0.16\columnwidth}\raggedright
\(t = 2\)\strut
\end{minipage} & \begin{minipage}[t]{0.19\columnwidth}\raggedright
\(y_{22}\)\strut
\end{minipage} & \begin{minipage}[t]{0.19\columnwidth}\raggedright
\(y_{12}\)\strut
\end{minipage} & \begin{minipage}[t]{0.35\columnwidth}\raggedright
\(y_{12} - y_{22}\)\strut
\end{minipage}\tabularnewline
\begin{minipage}[t]{0.16\columnwidth}\raggedright
\(t = 1\)\strut
\end{minipage} & \begin{minipage}[t]{0.19\columnwidth}\raggedright
\(y_{21}\)\strut
\end{minipage} & \begin{minipage}[t]{0.19\columnwidth}\raggedright
\(y_{11}\)\strut
\end{minipage} & \begin{minipage}[t]{0.35\columnwidth}\raggedright
\(y_{11} - y_{21}\)\strut
\end{minipage}\tabularnewline
\begin{minipage}[t]{0.16\columnwidth}\raggedright
Change\strut
\end{minipage} & \begin{minipage}[t]{0.19\columnwidth}\raggedright
\(y_{21} - y_{22}\)\strut
\end{minipage} & \begin{minipage}[t]{0.19\columnwidth}\raggedright
\(y_{11} - y_{22}\)\strut
\end{minipage} & \begin{minipage}[t]{0.35\columnwidth}\raggedright
\((y_{11} - y_{21}) - (y_{12} - y_{22})\)\strut
\end{minipage}\tabularnewline
\bottomrule
\end{longtable}

Effetto del trattamento:
\(\beta_3 = (y_{11} - y_{21}) - (y_{12} - y_{22})\)

\end{frame}

\begin{frame}{Table 3}
\protect\hypertarget{table-3}{}

\captionsetup[table]{labelformat=empty}
\begingroup\fontsize{6}{8}\selectfont

\begin{longtable}[t]{l>{\raggedleft\arraybackslash}p{5em}>{\raggedleft\arraybackslash}p{5em}>{\raggedleft\arraybackslash}p{5em}>{\raggedleft\arraybackslash}p{5em}>{\raggedleft\arraybackslash}p{5em}>{\raggedleft\arraybackslash}p{5em}}
\caption{\label{tab:unnamed-chunk-6}\textbf{Table 3 - Before and After Average Employement per Store}}\\
\toprule
\textbf{ } & \textbf{Mean FTE before} & \textbf{SE FTE before} & \textbf{Mean FTE after} & \textbf{SE FTE after} & \textbf{Change in mean} & \textbf{Change in mean, balanced sample}\\
\midrule
\addlinespace[0.3em]
\multicolumn{7}{l}{\textbf{Stores by state}}\\
\hspace{1em}PA & 23.33 & 0.59 & 21.17 & 0.41 & -2.17 & -2.28\\
\hspace{1em}NJ & 20.44 & 0.46 & 21.03 & 0.47 & 0.59 & 0.47\\
\hspace{1em}NJ - PA & -2.89 & 0.74 & -0.14 & 0.62 & 2.75 & 2.75\\
\addlinespace[0.3em]
\multicolumn{7}{l}{\textbf{Stores in NJ}}\\
\hspace{1em}Wage low & 19.56 & 0.43 & 20.88 & 0.56 & 1.32 & 1.20\\
\hspace{1em}Wage medium & 20.08 & 0.56 & 20.96 & 0.50 & 0.87 & 0.71\\
\hspace{1em}Wage high & 22.25 & 0.54 & 20.21 & 0.48 & -2.04 & -2.16\\
\addlinespace[0.3em]
\multicolumn{7}{l}{\textbf{Difference within NJ}}\\
\hspace{1em}Low - High & -2.69 & 0.69 & 0.66 & 0.74 & 3.36 & 3.36\\
\hspace{1em}Medium - High & -2.17 & 0.78 & 0.74 & 0.70 & 2.91 & 2.87\\
\bottomrule
\end{longtable}
\endgroup{}

FTE = numero lavoratori full-time + \(\frac{1}{2}\) numero lavoratori
part-time.

\end{frame}

\begin{frame}{Card \& Krueger (1994)}
\protect\hypertarget{card-krueger-1994}{}

\begin{itemize}
\item
  Il metodo difference in differences non tiene conto di altre possibili
  cause di variazione nell'occupazione.
\item
  Validiamo utilizzando altre regressioni lineari.
\end{itemize}

\[
\Delta E_i = \alpha + \beta X_i + \gamma NJ_i + \varepsilon_i
\] dove:

\begin{itemize}
\tightlist
\item
  \(\Delta E_i\) = variazione dell'occupazione fra \(t_1\) e \(t_2\) nel
  ristorante \(i\)
\item
  \(X_i\) = matrice di covariate
\item
  \(NJ_i\) = 1 se il ristorante è in New Jersey, 0 altrimenti
\end{itemize}

\end{frame}

\begin{frame}{Card \& Krueger (1994)}
\protect\hypertarget{card-krueger-1994-1}{}

\[
\Delta E_i = \alpha' + \beta' X_i + \gamma' GAP_i + \varepsilon_i
\] dove:

\begin{itemize}
\tightlist
\item
  \(GAP_i\) = 0 se il ristorante è in Pennsylvania
\item
  \(GAP_i\) = 0 se il ristorante è in New Jersey e \(W_{1i} = \$5.05\)
\item
  \(GAP_i\) = \((5.05 - W_{1i})/W_{1i}\) per gli altri ristoranti in New
  Jersey
\end{itemize}

\end{frame}

\begin{frame}{Table 4}
\protect\hypertarget{table-4}{}

\captionsetup[table]{labelformat=empty}
\begingroup\fontsize{7}{9}\selectfont

\begin{longtable}[t]{l>{\raggedright\arraybackslash}p{5em}>{\raggedright\arraybackslash}p{5em}>{\raggedright\arraybackslash}p{5em}>{\raggedright\arraybackslash}p{5em}>{\raggedright\arraybackslash}p{5em}>{}p{5em}}
\caption{\label{tab:unnamed-chunk-7}\textbf{Table 4}}\\
\toprule
\textbf{ } & \textbf{i} & \textbf{ii} & \textbf{iii} & \textbf{iv} & \textbf{v}\\
\midrule
Coefficient & 2.33 & 2.3 & 15.65 & 14.92 & 11.98\\
SE Coefficient & 1.19 & 1.2 & 6.08 & 6.21 & 7.42\\
SE Regression & 8.79 & 8.78 & 8.76 & 8.76 & 8.75\\
Prob. Controls & - & 0.34 & - & 0.44 & 0.4\\
\bottomrule
\end{longtable}
\endgroup{}

\begin{itemize}
\item
  \begin{enumerate}
  [i)]
  \tightlist
  \item
    New Jersey dummy.
  \end{enumerate}
\item
  \begin{enumerate}
  [i)]
  \setcounter{enumi}{1}
  \tightlist
  \item
    New Jersey dummy e chain dummy.
  \end{enumerate}
\item
  \begin{enumerate}
  [i)]
  \setcounter{enumi}{2}
  \tightlist
  \item
    GAP variable.
  \end{enumerate}
\item
  \begin{enumerate}
  [i)]
  \setcounter{enumi}{3}
  \tightlist
  \item
    GAP variable e chain dummy.
  \end{enumerate}
\item
  \begin{enumerate}
  [a)]
  \setcounter{enumi}{21}
  \tightlist
  \item
    GAP variable, chain dummy e regional dummy.
  \end{enumerate}
\end{itemize}

\end{frame}

\begin{frame}{Conclusioni}
\protect\hypertarget{conclusioni}{}

\begin{itemize}
\tightlist
\item
  L'aumento del salario minimo ha prodotto un aumento del tasso di
  occupazione.
\item
  Il risultato ottenuto con il metodo difference in differences è stato
  validato da regressione lineari, con o senza controlli.
\item
  Nell'articolo sono stati studiati altri outcome (frazione dei
  lavoratori full-time, ore di apertura dei ristoranti, tempo per il
  primo aumento salariale).
\end{itemize}

\end{frame}

\end{document}
